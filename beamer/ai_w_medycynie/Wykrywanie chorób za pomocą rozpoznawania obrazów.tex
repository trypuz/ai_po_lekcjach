
\documentclass{beamer}
\usepackage[T1]{fontenc}
\usepackage[polish]{babel}
\usepackage[utf8]{inputenc}
\usepackage{array}
\usepackage{url}
\usetheme{Boadilla}
\setbeamertemplate{frametitle continuation}{}
\titlegraphic{\includegraphics[width=4cm]{../logo_kul.jpeg}}        
\title{Wykrywanie chorób za pomocą rozpoznawania obrazów}
\author{KPI KUL}
\institute{\url{ai.kul.pl}}
\date{2023-04-14}

\begin{document}
\frame{\titlepage}

\AtBeginSection[]
{
    \begin{frame}[allowframebreaks]
        \frametitle{Spis treści}
        \tableofcontents[currentsection]
    \end{frame}
}    


\section{Przykłady za stosowań rozpoznawania obrazów w medycynie}

\begin{frame}[fragile]
\frametitle{Dermatologia}
 \begin{itemize}
\item Czy znamię melanocytowe to czerniak (nowotwór złośliwy skóry)?
\end{itemize}
\end{frame}


\end{document}
    